% Created 2025-05-08 dj. 23:38
% Intended LaTeX compiler: pdflatex
\documentclass[11pt]{article}
\usepackage[utf8]{inputenc}
\usepackage[T1]{fontenc}
\usepackage{graphicx}
\usepackage{longtable}
\usepackage{wrapfig}
\usepackage{rotating}
\usepackage[normalem]{ulem}
\usepackage{amsmath}
\usepackage{amssymb}
\usepackage{capt-of}
\usepackage{hyperref}
\usepackage{float}
\usepackage{geometry}
\usepackage{parskip}
\geometry{a4paper}
\author{Josep Bonet i Saez (1633723)}
\date{\today}
\title{Report Behaviour Tree}
\hypersetup{
 pdfauthor={Josep Bonet i Saez (1633723)},
 pdftitle={Report Behaviour Tree},
 pdfkeywords={},
 pdfsubject={},
 pdfcreator={Emacs 30.1 (Org mode 9.7.11)}, 
 pdflang={English}}
\begin{document}

\maketitle
\tableofcontents

\section{Intro}
\label{sec:org99374d3}
In this assignment, we will use Behaviour Trees to implement relatively complex behaviours within the AAPE platform. Our main characters will be the Astronaut and CritterMantaRay agents.

Different code files were given at the start. An example behaviour tree, BTRoam.py was provided as a reference example. Goals BT.py included three example goals (DoNothing, ForwardDist, and Turn) so we could test BTRoam.py file.

AAgent-1.json is the configuration file for the Astronaut, while AAgent-2.json is for
the Critter agent. You may modify these files as needed or create new ones.
Additionally, APackAstroCritters.json is an example configuration file for the
Spawner (Spawner.pt), which spawns one Astronaut and ten Critters.
The AAgent\textsubscript{BT.py} is the agent's main code. Initially, you only need to modify that file
to register your goals and behaviour trees.
\section{Scenario Alone}
\label{sec:org0401d27}
Our Astronaut has been assigned to the DarkHole deep-space collection outpost. Her mission is to collect a rare species of alien flower. Fortunately, she is alone at the outpost, so gathering the flowers should be straightforward.

The Astronaut starts at the Base outpost (spawn point 0). The number of alien flowers remains constant. Whenever the astronaut collects a flower, a new one will randomly spawn in the harvest zone after a short delay. Implement her behaviour.
\subsection{Behaviour Tree Alone}
\label{sec:org41db1b6}
The behaviour Tree is implemented in BTRoam.py:

Selector: Root
\begin{itemize}
\item Sequence: ReturnToBase --> BN\_ CheckInventoryFull; BN\_ ReturnToBase
\item Sequence: CollectFlower --> BN\_ DetectFlower; BN\_ MoveToFlower
\item Selector: Wander --> BN\_ Avoid; BN\_ RandomRoam
\end{itemize}

The following classes were implemented in Goals\_ BT.py
\begin{itemize}
\item RandomRoam: Single-step stochastic roaming
\begin{itemize}
\item If a flower is spotted: preempts
\item 20\% chance to turn slightly, 80\% chance to step forward
\item Returns true whenever it issues an action
\end{itemize}
\item Avoid: Single-tick obstacle avoidance
\begin{itemize}
\item If any Rock/wall is within threshold distance, turn away and return true
\item Return False so the tree moves on to RandomRoam
\end{itemize}
\item DetectFlower: Checks if the sensor find an "AlienFlower"
\item MoveToFlower:
\item ReturnToBase: Returnsto the base and then the saves the flowers.
\item CheckInventory: Counts the flowers in inventory and returns true if there are 2 or more.
\end{itemize}
\section{Scenario Critters}
\label{sec:org6dfd90e}
We now move to a new collection outpost, situated on a small moon near Saturn. However, hostile life forms have been discovered there. Now we will implement the behaviour of these life forms.
\subsection{Behaviour Tree Critters}
\label{sec:org9dda7be}
\section{Scenario Collect-and-run}
\label{sec:org3071b71}
\end{document}
